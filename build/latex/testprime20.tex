%% Generated by Sphinx.
\def\sphinxdocclass{report}
\documentclass[letterpaper,10pt,english]{sphinxmanual}
\ifdefined\pdfpxdimen
   \let\sphinxpxdimen\pdfpxdimen\else\newdimen\sphinxpxdimen
\fi \sphinxpxdimen=.75bp\relax
\ifdefined\pdfimageresolution
    \pdfimageresolution= \numexpr \dimexpr1in\relax/\sphinxpxdimen\relax
\fi
%% let collapsible pdf bookmarks panel have high depth per default
\PassOptionsToPackage{bookmarksdepth=5}{hyperref}

\PassOptionsToPackage{warn}{textcomp}
\usepackage[utf8]{inputenc}
\ifdefined\DeclareUnicodeCharacter
% support both utf8 and utf8x syntaxes
  \ifdefined\DeclareUnicodeCharacterAsOptional
    \def\sphinxDUC#1{\DeclareUnicodeCharacter{"#1}}
  \else
    \let\sphinxDUC\DeclareUnicodeCharacter
  \fi
  \sphinxDUC{00A0}{\nobreakspace}
  \sphinxDUC{2500}{\sphinxunichar{2500}}
  \sphinxDUC{2502}{\sphinxunichar{2502}}
  \sphinxDUC{2514}{\sphinxunichar{2514}}
  \sphinxDUC{251C}{\sphinxunichar{251C}}
  \sphinxDUC{2572}{\textbackslash}
\fi
\usepackage{cmap}
\usepackage[T1]{fontenc}
\usepackage{amsmath,amssymb,amstext}
\usepackage{babel}



\usepackage{tgtermes}
\usepackage{tgheros}
\renewcommand{\ttdefault}{txtt}



\usepackage[Bjarne]{fncychap}
\usepackage{sphinx}

\fvset{fontsize=auto}
\usepackage{geometry}


% Include hyperref last.
\usepackage{hyperref}
% Fix anchor placement for figures with captions.
\usepackage{hypcap}% it must be loaded after hyperref.
% Set up styles of URL: it should be placed after hyperref.
\urlstyle{same}

\addto\captionsenglish{\renewcommand{\contentsname}{Administrator Documentation}}

\usepackage{sphinxmessages}
\setcounter{tocdepth}{2}



\title{Test PrIME20 Documentation}
\date{May 13, 2022}
\release{1}
\author{sfa}
\newcommand{\sphinxlogo}{\vbox{}}
\renewcommand{\releasename}{Release}
\makeindex
\begin{document}

\pagestyle{empty}
\sphinxmaketitle
\pagestyle{plain}
\sphinxtableofcontents
\pagestyle{normal}
\phantomsection\label{\detokenize{index::doc}}


\sphinxstepscope


\chapter{Displaying Headings}
\label{\detokenize{topic1:displaying-headings}}\label{\detokenize{topic1::doc}}

\section{Heading 1}
\label{\detokenize{topic1:heading-1}}

\subsection{Sub\sphinxhyphen{}Heading 1}
\label{\detokenize{topic1:sub-heading-1}}

\section{Heading 2}
\label{\detokenize{topic1:heading-2}}

\section{Heading 3}
\label{\detokenize{topic1:heading-3}}
\sphinxAtStartPar
dkfla;sfsfasdfoasfvasfa

\sphinxAtStartPar
If extensions (or modules to document with autodoc) are in another directory,
\# add these directories to sys.path here. If the directory is relative to the
\# documentation root, use os.path.abspath to make it absolute, like shown here.
\#

\sphinxstepscope


\chapter{Displaying Images}
\label{\detokenize{topic2:displaying-images}}\label{\detokenize{topic2::doc}}

\section{Image 1}
\label{\detokenize{topic2:image-1}}
\noindent\sphinxincludegraphics{{lightbulb}.png}

\sphinxstepscope


\chapter{Linking Pages}
\label{\detokenize{topic3:linking-pages}}\label{\detokenize{topic3::doc}}

\section{Displaying Headings Page Linked}
\label{\detokenize{topic3:displaying-headings-page-linked}}
\sphinxAtStartPar
This is a link to {\hyperref[\detokenize{topic1::doc}]{\sphinxcrossref{\DUrole{doc}{Displaying Headings}}}}.

\sphinxAtStartPar
If extensions (or modules to document with autodoc) are in another directory,
\# add these directories to sys.path here. If the directory is relative to the
\# documentation root, use os.path.abspath to make it absolute, like shown here.
\#

\sphinxstepscope


\chapter{Topic3}
\label{\detokenize{topic4:topic3}}\label{\detokenize{topic4::doc}}
\sphinxstepscope


\chapter{FAQs}
\label{\detokenize{FAQs:faqs}}\label{\detokenize{FAQs::doc}}
\sphinxstepscope


\chapter{Release Notes}
\label{\detokenize{release_notes:release-notes}}\label{\detokenize{release_notes::doc}}

\chapter{Indices and tables}
\label{\detokenize{index:indices-and-tables}}\begin{itemize}
\item {} 
\sphinxAtStartPar
\DUrole{xref,std,std-ref}{genindex}

\item {} 
\sphinxAtStartPar
\DUrole{xref,std,std-ref}{modindex}

\item {} 
\sphinxAtStartPar
\DUrole{xref,std,std-ref}{search}

\end{itemize}



\renewcommand{\indexname}{Index}
\printindex
\end{document}